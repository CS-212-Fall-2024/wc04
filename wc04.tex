\documentclass[a4paper]{exam}

\usepackage{amsmath,amssymb, amsthm}
\usepackage[a4paper]{geometry}
\usepackage{hyperref}
\usepackage{mdframed}
\usepackage{arabtex}
\usepackage{utf8}
\setcode{utf8}


\title{Weekly Challenge 04: Regular Expressions}
\author{CS 212 Nature of Computation\\Habib University}
\date{Fall 2024}

\theoremstyle{definition}
\newtheorem{definition}{Definition}

\theoremstyle{claim}
\newtheorem{claim}{Claim}

\qformat{{\large\bf \thequestion. \thequestiontitle}\hfill}
\boxedpoints

% \printanswers %uncomment this

\begin{document}
\maketitle

\begin{questions}
    \titledquestion{Closures}
    Given a language, $L$, and the definitions below, prove or disprove the given claim.

    \begin{definition}[Power on a language]
        Given a language $L$, we define $L^n$, for $n \in \mathbb{N}$ recursively as follows:
        \begin{enumerate}
            \item $L^0 = \{\varepsilon\}$ 
            \item $L^1 = L$
            \item $L^{k+1} = \{w_1w_2\mid w_1 \in L^k \land w_2 \in L\}$
        \end{enumerate}
    \end{definition}
    \begin{definition}[Kleene closure]
        With this definition of $L^n$, we can define Kleene closure of $L$, $L^*$ as follows:
        $$L^*=\bigcup_{i \in \mathbb{N}} L^i$$
    \end{definition}
    \begin{definition}[Positive closure]
        With this definition of $L^n$, we can define Positive closure of $L$, $L^+$ as follows:
        $$L^+=\bigcup_{i \in \mathbb{Z}^+} L^i$$
    \end{definition}
    \begin{claim}
      $(L^+)^*=(L^*)^+$
    \end{claim}  
    
    \textbf{Note:} $\mathbb{N}$ here denotes the set of natural numbers which includes 0. See \href{https://en.wikipedia.org/wiki/Set-theoretic_definition_of_natural_numbers}{von Neumann ordinals} for the construction of $\mathbb{N}$.
    $\mathbb{Z}^+$ denotes the set of positive integers.
    % Enter your solution below
  \begin{solution}
    
  \end{solution}
  
\end{questions}
\end{document}

%%% Local Variables:
%%% mode: latex
%%% TeX-master: t
%%% End:
